\documentclass[a4paper,12pt]{article} % Set paper size and document type
\usepackage{amsmath,eqparbox}

% Change margins - default margins are too broad
\usepackage[margin=10mm]{geometry}

% Set document title and author
\title{ProblemSet2 Math For AI}
\author{Haidun LI}
\date{2023-09-29} % If date is left blank, it will be hidden

\begin{document}
\noindent
Hello world! $2^2 = \sqrt{16}$

\noindent
\textbf{1.Translate the following system of 3 equations and 3 unknowns into an augmented matrix format and solve for the unknowns} \\

\vspace{2mm}

\begin{align*}
\begin{cases}
 	x + y + z & =  6 \\
	2x - y + z & = -3 \\
	3y - 2z & = 0 \\
\end{cases}
\end{align*}

convert this linear algebra equation to a matrix: using Gaussian elimination
\[
\begin{bmatrix}
    1 & 1 & 1 & | & 6 \\
    2 & -1 & 1 & | & -3 \\
    0 & 3 & -2 & | & 0 \\
\end{bmatrix}
\]

R2 = -2R1 + R2:
\[
\begin{bmatrix}
    1 & 1 & 1 & | & 6 \\
    0 & -3 & -1 & | & -15 \\
    0 & 3 & -2 & | & 0 \\
\end{bmatrix}
\]

R3 = R3 + R2
\[
\begin{bmatrix}
    1 & 1 & 1 & | & 6 \\
    0 & -3 & -1 & | & -15 \\
    0 & 0 & -3 & | & -15 \\
\end{bmatrix}
\] \\

Now, we have solve z parameter, which value is: \\
\begin{center}
    -3z = -15 \\  z =5
\end{center}

Then, the linear algebra equation will be like:

\begin{align*}
\begin{cases}
 	x + y + 5 & =  6 \\
	- 3y - 5 & = -15 \\
\end{cases}
\end{align*}

So, the y parameter value is: \\
\begin{center}
    -3y -5 = -15 \\ -3y = -10 \\ y = $\frac{10}{3}$
\end{center}

And then, we could use parameter y, z to solve x:
\begin{center}
x+$\frac{10}{3}$ +5 = 6 \\
x = -$\frac{7}{3}$
\end{center}

As a result,  the solution to the system of equations is:
\begin{center}
x = -$\frac{7}{3}$, y = $\frac{10}{3}$, z=5
\end{center}

\vspace{2mm}



\noindent
\textbf{2. Identify which, if any, of the following 2x2 matrices are singular} \\
\\2.1: \\
\begin{align*}
\begin{pmatrix}
    2 & 6 \\
    3 & -2 \\
\end{pmatrix}
\end{align*}
This determinant of this matrix is: \textbf{NOT} singular
\begin{align*}
\text{det}(A) &= (2 \cdot (-2)) - (6 \cdot 3) \\
&= (-4) - (18) \\
&= -22 \\
-22 &\neq 0
\end{align*}
\\
2.2: \\
\begin{align*}
\begin{pmatrix}
    1 & 11 \\
    3 & 40 \\
\end{pmatrix}
\end{align*}
This determinant of this matrix is: \textbf{NOT} singular
\begin{align*}
\text{det}(A) &= (1 \cdot (40)) - (11 \cdot 3) \\
&= 40 - 33 \\
&= 7 \\
7 &\neq 0
\end{align*}
\\
2.3: \\
\begin{align*}
\begin{pmatrix}
    22 & 11 \\
    8 & 4 \\
\end{pmatrix}
\end{align*}
This determinant of this matrix is: \textbf{singular}
\begin{align*}
\text{det}(A) &= (22 \cdot (4)) - (8 \cdot 11) \\
&= 88 - 88 \\
&= 0 \\
\end{align*}
\\
2.4: \\
\begin{align*}
\begin{pmatrix}
    d^2 & d+c \\
    d & c/d+1 \\
\end{pmatrix}
\end{align*}
This determinant of this matrix is: \textbf{singular}
\begin{align*}
\text{det}(A) &= (d^2 \cdot (c/d+1)) - ((d+c) \cdot d) \\
&= cd+d^2 - d^2-cd  \\
&= 0 \\
\end{align*}
2.5: \\
\begin{align*}
\begin{pmatrix}
    0.5 & 5 \\
    2 & 22 \\
\end{pmatrix}
\end{align*}
This determinant of this matrix is: \textbf{NOT} singular
\begin{align*}
\text{det}(A) &= (0.5 \cdot 22) - (5 \cdot 2) \\
&= 11 - 10 \\
&= 1 \\
1 &\neq 0
\end{align*}
\\

\noindent
\textbf{3. Find the determinant of each of the following 3x3 matrices.} \\
\\ 3.1: \\
\[
\begin{bmatrix}
    1 & 2 & 3 \\
    -1 & 5 & 7 \\
    4 & 5 & 2 \\
\end{bmatrix}
\]
The determinant of this matrix is:
\begin{align*}
\text{det}(A) &= (1 \cdot (5 \cdot 2 - 7 \cdot 5)) - (2 \cdot (-1 \cdot 2 - 7 \cdot 4)) + (3 \cdot (-1 \cdot 5 - 5 \cdot 4)) \\
&= (1 \cdot (-25)) - (2 \cdot (-30)) + (3 \cdot (-25)) \\
&= -25 + 60 - 75 \\
&= -40 \\
\end{align*}
\\

3.2: \\
\[
\begin{bmatrix}
    6 & 2 & -1 \\
    -4 & 1/5 & 0 \\
    6 & 10 & -5 \\
\end{bmatrix}
\]
The determinant of this matrix is:
\begin{align*}
\text{det}(A) &= (6 \cdot (1/5 \cdot (-5) - 0 \cdot 10)) - (2 \cdot (-4 \cdot (-5) - 0 \cdot 6)) + (-1 \cdot (-4 \cdot 10 - 1/5 \cdot 6)) \\
&= (6 \cdot (-1)) - (2 \cdot 20) + (-1 \cdot (-40)) \\
&= -6 - 40 + 40 \\
&= -6 \\
\end{align*}
\\

3.3: \\
\[
\begin{bmatrix}
    3 & -1 & 2 \\
    5 & 1 & 0 \\
    -2 & 3 & 4 \\
\end{bmatrix}
\]
The determinant of this matrix is:
\begin{align*}
\text{det}(A) &= (3 \cdot (1 \cdot 4 - 0 \cdot 3)) - (-1 \cdot (5 \cdot 4 - 0 \cdot (-2))) + (2 \cdot (5 \cdot 3 - 1 \cdot (-2))) \\
&= (3 \cdot 4) - (-1 \cdot 22) + (2 \cdot 17) \\
&= 12 + 22 + 34 \\
&= 68 \\
\end{align*}
\\

\noindent
\textbf{4. Find the inverse of the following 3x3 matrices. You can use any method you like, but it would be good practice to try 2 different methods.} \\
\\ 4.1: \\
\[
\begin{bmatrix}
    2 & 0 & -1 \\
    5 & 1 & 0 \\
    0 & 1 & 3 \\
\end{bmatrix}
\] \\
\\
The process of finding the inverse of this matrix is to find the determinant of this matrix first: \\
\begin{align*}
\text{det}(A) &= (2 \cdot (1 \cdot 3 - 0 \cdot 1)) - (0 \cdot (5 \cdot 3 - 0 \cdot 0)) + (-1 \cdot (5 \cdot 1 - 1 \cdot 0)) \\
&= (2 \cdot 3) - (0 \cdot 15) + (-1 \cdot 5) \\
&= 6 - 5 \\
&= 1 \\
\end{align*}
\noindent
Transpose of this matrix is: \\
\[
\begin{bmatrix}
    2 & 5 & 0 \\
    0 & 1 & 1 \\
    -1 & 0 & 3 \\
\end{bmatrix}
\]
\\
\noindent
The Matrix R1C1 is:
\[
\begin{pmatrix}
    1 & 1 \\
    0 & 3 \\
\end{pmatrix}
\]
And the value of this matrix is: 3 \\\\
\\
\noindent
The Matrix R1C2 is:
\[
\begin{pmatrix}
    0 & 1 \\
    -1 & 3 \\
\end{pmatrix}
\]
And the value of this matrix is: 1 \\\\
\noindent
The Matrix R1C3 is:
\[
\begin{pmatrix}
    0 & 1 \\
    1 & 0 \\
\end{pmatrix}
\]
And the value of this matrix is: -1 \\\\

\noindent
The Matrix R2C1 is:
\[
\begin{pmatrix}
    5 & 0 \\
    0 & 3 \\
\end{pmatrix}
\]
And the value of this matrix is: 15 \\\\

\noindent
The Matrix R2C2 is:
\[
\begin{pmatrix}
    2 & 0 \\
    -1 & 3 \\
\end{pmatrix}
\]
And the value of this matrix is: 6 \\\\

\noindent
The Matrix R2C3 is:
\[
\begin{pmatrix}
    2 & 5 \\
    -1 & 0 \\
\end{pmatrix}
\]
And the value of this matrix is: -5 \\\\

\noindent
The Matrix R3C1 is:
\[
\begin{pmatrix}
    5 & 0 \\
    1 & 1 \\
\end{pmatrix}
\]
And the value of this matrix is: 5 \\\\

\noindent
The Matrix R3C2 is:
\[
\begin{pmatrix}
    2 & 0 \\
    0 & 1 \\
\end{pmatrix}
\]
And the value of this matrix is: 2 \\\\

\noindent
The Matrix R3C3 is:
\[
\begin{pmatrix}
    2 & 5 \\
    0 & 1 \\
\end{pmatrix}
\]
And the value of this matrix is: 2 \\\\

\noindent
Collect all of these value together, we could get the temporary matrix: \\
\[
\begin{bmatrix}
    3 & 1 & -1 \\
    15 & 6 & -5 \\
    5 & 2 & 2 \\
\end{bmatrix}
\]
\\
\noindent
adjoint matrix: change the temporary matrix each value's symbol: \\
\[
\begin{bmatrix}
    3 & -1 & 1 \\
    -15 & 6 & 5 \\
    5 & -2 & 2 \\
\end{bmatrix}
\]
Using adjoint matrix divide the determinant of this matrix: \\
And the inverse of this matrix is: \\
\[
\begin{bmatrix}
    3 & -1 & 1 \\
    -15 & 6 & 5 \\
    5 & -2 & 2 \\
\end{bmatrix}
\]
\\



\noindent
\\Using Gaussian elimination to find the inverse of this matrix: \\
\[
\begin{bmatrix}
    2 & 0 & -1 & | & 1 & 0 & 0 \\
    5 & 1 & 0 & | & 0 & 1 & 0 \\
    0 & 1 & 3 & | & 0 & 0 & 1 \\
\end{bmatrix}
\]
\\
R1 = R1/2
\[
\begin{bmatrix}
    1 & 0 & -1/2 & | & 1/2 & 0 & 0 \\
    5 & 1 & 0 & | & 0 & 1 & 0 \\
    0 & 1 & 3 & | & 0 & 0 & 1 \\
\end{bmatrix}
\]
\\
R1 = 5R1
\[
\begin{bmatrix}
    5 & 0 & -5/2 & | & 5/2 & 0 & 0 \\
    5 & 1 & 0 & | & 0 & 1 & 0 \\
    0 & 1 & 3 & | & 0 & 0 & 1 \\
\end{bmatrix}
\]
\\
R2 = R2 - R1
\[
\begin{bmatrix}
    5 & 0 & -5/2 & | & 5/2 & 0 & 0 \\
    0 & 1 & 5/2 & | & -5/2 & 1 & 0 \\
    0 & 1 & 3 & | & 0 & 0 & 1 \\
\end{bmatrix}
\]
\\
R1 = R1/5
\[
\begin{bmatrix}
    1 & 0 & -1/2 & | & 1/2 & 0 & 0 \\
    0 & 1 & 5/2 & | & -5/2 & 1 & 0 \\
    0 & 1 & 3 & | & 0 & 0 & 1 \\
\end{bmatrix}
\]
\\
R3 = R3 - R2
\[
\begin{bmatrix}
    1 & 0 & -1/2 & | & 1/2 & 0 & 0 \\
    0 & 1 & 5/2 & | & -5/2 & 1 & 0 \\
    0 & 0 & 1/2 & | & 5/2 & -1 & 1 \\
\end{bmatrix}
\]
\\
R3 = 2R3
\[
\begin{bmatrix}
    1 & 0 & -1/2 & | & 1/2 & 0 & 0 \\
    0 & 1 & 5/2 & | & -5/2 & 1 & 0 \\
    0 & 0 & 1 & | & 5 & -2 & 2 \\
\end{bmatrix}
\]
\\
R3 = R3 * -1/2
\[
\begin{bmatrix}
    1 & 0 & -1/2 & | & 1/2 & 0 & 0 \\
    0 & 1 & 5/2 & | & -5/2 & 1 & 0 \\
    0 & 0 & -1/2 & | & -5/2 & 1 & -1 \\
\end{bmatrix}
\]
\\
R1 = R1 - R3
\[
\begin{bmatrix}
    1 & 0 & 0 & | & 3 & -1 & 1 \\
    0 & 1 & 5/2 & | & -5/2 & 1 & 0 \\
    0 & 0 & -1/2 & | & -5/2 & 1 & -1 \\
\end{bmatrix}
\]
\\
R3 = R3 * -5
\[
\begin{bmatrix}
    1 & 0 & 0 & | & 3 & -1 & 1 \\
    0 & 1 & 5/2 & | & -5/2 & 1 & 0 \\
    0 & 0 & 5/2 & | & 25/2 & -5 & 5 \\
\end{bmatrix}
\]
\\
R2 = R2 - R3
\[
\begin{bmatrix}
    1 & 0 & 0 & | & 3 & -1 & 1 \\
    0 & 1 & 0 & | & -15 & 6 & -5 \\
    0 & 0 & 5/2 & | & 25/2 & -5 & 5 \\
\end{bmatrix}
\]
\\
R3 = R3 * 2/5
\[
\begin{bmatrix}
    1 & 0 & 0 & | & 3 & -1 & 1 \\
    0 & 1 & 0 & | & -15 & 6 & -5 \\
    0 & 0 & 1 & | & 5 & -2 & 2 \\
\end{bmatrix}
\]
\\
As a result, the inverse of this matrix is: \\
\[
\begin{bmatrix}
    3 & -1 & 1 \\
    -15 & 6 & -5 \\
    5 & -2 & 2 \\
\end{bmatrix}
\]
\\

\noindent
4.2: \\
\[
\begin{bmatrix}
    3 & -1 & 2 \\
    5 & 1 & 0 \\
    -2 & 3 & 4 \\
\end{bmatrix}
\] \\
\\
The inverse of this matrix is: \\
\[
\begin{bmatrix}
    2/33 & 5/33 & -1/33 \\
    -10/33 & 8/33 & 5/33 \\
    17/66 & -7/66 & 4/33 \\
\end{bmatrix}
\]
\\


\noindent
\textbf{5. Write down the transpose of each of the following matrices.} \\
5.1: \\
\[
\begin{bmatrix}
    5 & -55 & 6 & 5 \\
    12 & 1/2 & -7 & 22 \\
\end{bmatrix}
\]
The transpose of this matrix is: \\
\[
\begin{bmatrix}
    5 & 12 \\
    -55 & 1/2 \\
    6 & -7 \\
    5 & 22 \\
\end{bmatrix}
\]
\\
5.2: \\
\[
\begin{bmatrix}
    1 & 2 & 3 & 4 \\
    0 & 2 & 4 & 9 \\
    12 & 20 & -1 & 11 \\
    -2 & -15 & 0 & 13 \\
\end{bmatrix}
\]
The transpose of this matrix is: \\
\[
\begin{bmatrix}
    1 & 0 & 12 & -2 \\
    2 & 2 & 20 & -15 \\
    3 & 4 & -1 & 0 \\
    4 & 9 & 11 & 13 \\
\end{bmatrix}
\]
\\








\end{document}